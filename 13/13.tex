\documentclass{bmstu}

\usepackage{mathtools}

\usepackage{physics}
\usepackage{pdfpages}
\usepackage{tabularx}
\usepackage{longtable}
\usepackage{xfrac}
\usepackage{amssymb}
\usepackage{dsfont}
\usepackage{upgreek}
\usepackage{color, colortbl}
\usepackage{listings}
\usepackage{ amssymb }
\usepackage{tikz}
\usetikzlibrary{decorations.markings}
\usetikzlibrary{calc} 
\usepackage{ wasysym }

\begin{document}
	\section*{06/12}
	\begin{center}
		\textbf{Сохранение непрерывности вдоль границ между элементами}
	\end{center}

\[
\begin{cases}
	\varphi^{(1)}=N_1^{(1)}\Phi_1+N_2^{(1)}\Phi_2+N_3^{(1)}\Phi_3+N_4^{(1)}\Phi_4 \\
	\varphi^{(2)}=N_1^{(2)}\Phi_5+N_2^{(2)}\Phi_6+N_3^{(2)}\Phi_2+N_4^{(2)}\Phi_1
\end{cases}
\]
\[
	\begin{cases}
	N_1=\frac{1}{4}(1-\xi)(1-\eta) \\
	N_2=\frac{1}{4}(1+\xi)(1-\eta) \\
	N_3=\frac{1}{4}(1+\xi)(1+\eta) \\
	N_4=\frac{1}{4}(1-\xi)(1+\eta)
	\end{cases} 
\]
\[
 \eta^{(1)}=-1, \eta^{(2)}=1  \Rightarrow  \begin{cases}
 	\varphi^{(1)}=\frac{1}{2}(1-\xi^{(1)})\Phi_1 + \frac{1}{2}(1+\xi^{(1)})\Phi_2 \\
 	\varphi^{(2)}=\frac{1}{2}(1+\xi^{(2)})\Phi_2 + \frac{1}{2}(1-\xi^{(2)})\Phi_1
 \end{cases}
\]
\[
	\xi^{(1)}=\xi^{(2)} \Rightarrow \varphi^{(1)}=\varphi^{(2)}
\]

\begin{center}
	\textbf{Квадратичные и кубические четырехугольные КЭ из Серендипова семейства}
\end{center}
\[
\begin{cases}
\varphi_2=\alpha_1+\alpha_2x+\alpha_3y+\alpha_4xy+\alpha_5x^2y+\alpha_6xy^2+\alpha_7x^2+\alpha_8y^2 \\
\varphi_3= -//- +\alpha_9x^3+\alpha_{10}y^3+\alpha_{11}x^3y+\alpha_{12}xy^3
\end{cases}
\]
\[
\Rightarrow
\begin{cases}
	N^{(2)}=(\alpha_1+\alpha_2\xi+\alpha_3\eta+\alpha_4\xi\eta)(a_1+a_2\xi+a_3\eta) \\
	N^{(3)}=(\alpha_1+\alpha_2\xi+\alpha_3\eta+\alpha_4\xi\eta)(a_1+a_2\xi+a_3\eta+a_4\xi^2+a_5\eta^2)
\end{cases}
\]
Вычислим функцию формы для элемента 1:
\[
	N_1=(1-\eta)(1-\xi)(a_1+a_2\xi+a_3\eta) 
\]
\[ \begin{cases}
	N_1(\xi=0, \eta=1)=0 \\
	N_1(\xi=-1, \eta=0) =0 \\
	N_1(\xi=-1, \eta=1) = 1 \end{cases} \Rightarrow \begin{cases} N_1=2(a_1-a_3)=0 \\ N_1=2(a_1-a_2)=0 \\ N_1 = 4(a_1-a_2-a_3)=1  \end{cases} \Rightarrow a_1=a_2=a_3=-\frac{1}{4} 
\]
\[
	\Rightarrow N_1=-\frac{1}{4}(1-\eta)(1-\xi)(1+\xi+\eta) 
\]

Вычислим функцию формы для элемента 2:
\[
N_2=(\alpha_1+\alpha_2\xi+\alpha_3\eta+\alpha_4\xi\eta)(a_1+a_2\xi+a_3\eta)=(1+\xi)(1-\xi)(1-\eta)\cdot a \Rightarrow
\]
\[
	\Rightarrow 
	N_2(\xi=0, \eta=-1)=2a=1\Rightarrow a=\frac{1}{2} \Rightarrow N_2=\frac{1}{2}(1-\xi^2)(1-\eta)
\]

Для кубического элемента:
\[
N_1=(1-\xi)(1-\eta)(a_1+a_2\xi+a_3\eta+a_4\xi^2+a_5\eta^2)
\]
\[
\begin{cases}
	N_1(-\frac{1}{3},-1)=0  \\
	N_1(\frac{1}{3},-1)=0 \\
	N_1(-1,\frac{1}{3})=0 \\
	N_1(-1, -\frac{1}{3})=0 \\
	N_1(-1,-1)=1
\end{cases}
\]

\begin{center}
	\textbf{Вычисление производных}
\end{center}
\[
\begin{bmatrix}
	\frac{\partial N_{\beta}}{\partial \xi} \\
	\frac{\partial N_{\beta}}{\partial \eta}
\end{bmatrix} = \begin{bmatrix}
\frac{\partial x}{\partial \xi} & \frac{\partial x}{\partial \eta} \\
\frac{\partial y}{\partial \xi} & \frac{\partial y}{\partial \eta}
\end{bmatrix} \begin{bmatrix}
\frac{\partial N_{\beta}}{\partial x} \\
\frac{\partial N_{\beta}}{\partial y}
\end{bmatrix} \Rightarrow \begin{bmatrix}
\frac{\partial N_{\beta}}{\partial x} \\
\frac{\partial N_{\beta}}{\partial y}
\end{bmatrix}=J^{-1} \begin{bmatrix}
\frac{\partial N_{\beta}}{\partial \xi} \\
\frac{\partial N_{\beta}}{\partial \eta}
\end{bmatrix}
\]

Допустим, $\varphi_2=N_1\Phi_1+\dots+N_8\Phi_8$
\[
\begin{cases}
	x=R_1X_1+R_2X_2+R_3X_3+R_4X_4 \\
	y=R_1Y_1+R_2Y_2+R_3Y_3+R_4Y_4
\end{cases}  \text{ -- субпараметрический КЭ}
\]

где $R_i$ - линейные интерполяции

\[
\begin{cases}
	R_1=\frac{1}{4}(1-\xi)(1-\eta) \\
	R_2=\frac{1}{4}(1+\xi)(1-\eta) \\
	R_3=\frac{1}{4}(1+\xi)(1+\eta) \\
	R_4=\frac{1}{4}(1-\xi)(1+\eta)
\end{cases}
\]

Изопараметрический КЭ:
\[
\begin{cases}
	x=N_1X_1+\dots+N_8X_8 \\
	y=N_1Y_1+\dots+N_8Y_8
\end{cases}
\]

Так как все стороны линейные, интегралы можно свести к виду:
\[
\int\limits_VB^TDB\ dV=t\int\limits_{-1}^1\int\limits_{-1}^1 B^TDB |J|\ d\eta d\xi
\]
\[
Z=\int\limits_{-1}^1\int\limits_{-1}^1 f(\xi, \eta) \ d\eta d\xi = \sum\limits_{i=1}^{n}\sum\limits_{j=1}^{n}H_iH_j f(\xi_i, \eta_j)
\]
\[
\int\limits_{-1}^1\int\limits_{-1}^1 f(\xi, \eta) \ d\eta =\sum\limits_{j=1}^{n} H_j f (\xi, \eta_j) = g(\xi)
\]
\[
\int\limits_{-1}^1 g(\xi)\ d\xi = \sum\limits_{i=1}^{n} H_i g(\xi_i)
\]

\begin{center}
	\textbf{Лагранжево семейство}
\end{center}

Функция формы:
\[
N_{ij}=L_i^n(\xi)L_j^m(\eta)
\]

$L_i^n(\xi)L_j^m(\eta)$ - многочлены Лагранжа, $n,m$ - количество разбиений по $\xi, \eta$

\[
L_i^n(\xi)=\frac{(\xi-\xi_1)(\xi-\xi_2)\dots(\xi-\xi_n)}{(\xi_i-\xi_1)(\xi_i-\xi_2)\dots(\xi_i-\xi_n)}
\]
\[
N_{ij}=L_i^2(\xi)L_j^2(\eta),\ i \neq 1,2
\]
\[
L_i^2(\xi)=\frac{(\xi-\xi_1)(\xi-\xi_2)}{(\xi_i-\xi_1)(\xi_i-\xi_2)}
\]
\[
N_{11}=\frac{\xi\cdot(\xi-1)}{-1\cdot(-2)}\cdot \frac{\eta\cdot(\eta-1)}{-1\cdot(-2)}=\frac{1}{4}\cdot\xi\cdot\eta (\xi-1)(\eta-1)
\]
\[
N_{12}=\frac{\xi\cdot(\xi-1)}{2}\cdot \frac{(\eta+1)(\eta-1)}{1\cdot(-1)}
\]

\end{document}