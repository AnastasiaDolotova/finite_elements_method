\documentclass{bmstu}


\usepackage{physics}
\usepackage{pdfpages}
\usepackage{tabularx}
\usepackage{longtable}
\usepackage{xfrac}
\usepackage{amssymb}
\usepackage{dsfont}
\usepackage{upgreek}
\usepackage{color, colortbl}
\usepackage{listings}
\usepackage{ amssymb }
\usepackage{tikz}

\graphicspath{
	{graphics/}
}

\begin{document}
	
	\section*{25/10}
	\begin{center}
		\textbf{пупупу}
	\end{center}
	
	
	Связь между координатами:
	\[
	\begin{cases}
		L_i+L_j+L_k=1 \\
		x=L_ix_i+L_jx_j+L_kx_k \\
		y=L_iy_i+L_jy_j+L_ky_k
	\end{cases}
	\]
	
	$V=hdS, h=const-\text{толщина элемента}$
	
	\[
	h\iint\limits_Sf(x,y)\ dxdy= h\int\limits_0^1\int\limits_0^{1-L_j} f(L_i,L_j,L_k)|J|\ dL_iL_j
	\]
	\[
	J=\begin{bmatrix}
		\dfrac{\partial x}{\partial L_i} & \dfrac{\partial x}{\partial L_j} \\
		
		\dfrac{\partial y}{\partial L_i} & \dfrac{\partial y}{\partial L_j}
	\end{bmatrix} = \begin{bmatrix}
	x_i-x_k & x_j-x_k \\
	y_i-y_k & y_j-y_k
	\end{bmatrix} \Rightarrow |J|=\Delta
	\]
	
	
	Интегральные формулы, упрощающие вычисления:
	\[
	\int\limits_{\Gamma_{ij}} L_i^{\alpha}L_j^{\beta} \ d\Gamma = \frac{\alpha!\beta!}{(\alpha+\beta+1)!}\cdot l_{ij}
	\]
	\[
		\int\limits_{S} L_i^{\alpha}L_j^{\beta}L_k^{\gamma} \ dS = \frac{\alpha!\beta!\gamma!}{(\alpha+\beta+\gamma+2)!}\cdot 2S \tag{1}
	\]
	
	
	Возвращаемся к формуле из прошлой лекции:
	\[
	\int\limits_{\Omega}  N^T N\ dx dy = \int\limits_{\Omega}  
	\begin{bmatrix}
		N_iN_i & N_iN_j & N_iN_k \\
		N_iN_j & N_jN_j & N_jN_k \\
		N_iN_k & N_jN_k & N_kN_k \\
	\end{bmatrix}
	\ dx dy = 
	\]
	\[
	= \int\limits_S \begin{bmatrix}
		L_iL_i & L_iL_j & L_iL_k \\
		L_iL_j & L_jL_j & L_jL_k \\
		L_iL_k & L_jL_k & L_kL_k \\
	\end{bmatrix} |J| \ dL_i L_j = \dfrac{S}{12} \begin{bmatrix}
	2 & 1 & 1 \\ 1 & 2 & 1 \\ 1 & 1 & 2
	\end{bmatrix}
	\]
	
	Вычисляем компоненты матрицы по формуле (1):
	\[
	\int\limits_{\Omega} L_i^2 \ dS = \int\limits_{\Omega} L_i^2L_j^0L_k^0 \ dS=  \dfrac{2!\, 0!\, 0!}{(2+0+0+2)!}\cdot 2S=\frac{S}{6}
	\]
	\[
	\int\limits_SN^{\text{T}} \ dS = \int\limits_S \begin{bmatrix}
		N_i \\ N_j \\ N_k
	\end{bmatrix} \, dxdy = \int\limits_S \begin{bmatrix}
	L_i \\ L_j \\ L_k
	\end{bmatrix} \, dS = \frac{S}{3} \begin{bmatrix}
	1 \\ 1 \\ 1
	\end{bmatrix}
	\]
	
	\[
	\int\limits_{\Gamma} \left(K_x \cdot  \frac{\partial u}{\partial x} \cdot l_x + K_y \cdot  \frac{\partial u}{\partial y} \cdot l_y \right)v\ d\Gamma 
	\]
	\begin{enumerate}
		\item или $K_x \cdot  \frac{\partial u}{\partial x} \cdot l_x + K_y \cdot  \frac{\partial u}{\partial y} \cdot l_y=\hat\sigma$ или $q$ 
		\item или $K_x \cdot  \frac{\partial u}{\partial x} \cdot l_x + K_y \cdot  \frac{\partial u}{\partial y} \cdot l_y=-\alpha_g(u-\hat u)$ 
	\end{enumerate}
	
	Рассмотрим подробнее:
	\begin{enumerate}
		\item $
		\int\limits_{\Gamma} \hat \sigma \ d\Gamma = \{\delta \Phi\}^T \int\limits_{\Gamma} \hat \sigma N^T\ d\Gamma\int\limits_{\Gamma} \begin{bmatrix}
			L_i \\ L_j \\ L_k
		\end{bmatrix} \, d\Gamma$
		\begin{enumerate}
			\item $\int\limits_{\Gamma_{ij}} \begin{bmatrix}
				L_i \\ L_j \\ 0
			\end{bmatrix} \, d\Gamma = \int\limits_{\Gamma_{ij}} L_i^1 L_j^0 d\Gamma = \dfrac{l_{ij}}{2}\begin{bmatrix}
			1\\1\\0
		\end{bmatrix}$
		\item 
		\end{enumerate}
	\end{enumerate}
	
\end{document}