\documentclass{bmstu}

\usepackage{mathtools}

\usepackage{physics}
\usepackage{pdfpages}
\usepackage{tabularx}
\usepackage{longtable}
\usepackage{xfrac}
\usepackage{amssymb}
\usepackage{dsfont}
\usepackage{upgreek}
\usepackage{color, colortbl}
\usepackage{listings}
\usepackage{ amssymb }
\usepackage{tikz}
\usetikzlibrary{decorations.markings}
\usetikzlibrary{calc} 
\usepackage{ wasysym }
\usepackage{makecell}

\begin{document}
	\section*{29/11}
	\begin{center}
	\textbf{Четырехугольные конечные элементы}
	
	\textbf{Мультиплекс-элементы}
	\end{center}
	
	\begin{center}
		\begin{tikzpicture}
			\draw[->] (0,0) -- (6,0) node[below right] {$x$};
			\draw[->] (0,0) -- (0,5) node[below left] {$y$};
			\draw (1,1) rectangle ++(4,3);
			\filldraw (1,1) circle (1pt) node[left] {4};
			\filldraw (1,2) circle (1pt) node[left] {3};
			\filldraw (1,3) circle (1pt) node[left] {2};
			\filldraw (1,4) circle (1pt) node[left] {1};
			\filldraw (5,1) circle (1pt) node[right] {8};
			\filldraw (5,2) circle (1pt) node[right] {7};
			\filldraw (5,3) circle (1pt) node[right] {6};
			\filldraw (5,4) circle (1pt) node[right] {5};
		\end{tikzpicture}
	\end{center}

		$\varphi$ - аппроксимирующая функция должна быть непрерывной между элементами. 
		
		Предположим, что вдоль верхней и нижней сторон функция меняется по линейному закону, а вдоль вертикальных сторон, например, по кубическому.
		\[
		\varphi=\alpha_1+\alpha_2x+\alpha_3y+\alpha_4xy+\alpha_5y^2+\alpha_6xy^2+\alpha_7y^3+\alpha_8xy^3
		\]
		\[
		\underbrace{\begin{bmatrix}
			1 & x_1 & y_1 & x_1y_1 & y_1^2 & x_1y_1^2 & y_1^3 & x_1y_1^3 \\
			1 & x_2 & y_2 & x_2y_2 & y_2^2 & x_2y_2^2 & y_2^3 & x_2y_2^3 \\
			1 & x_3 & y_3 & x_3y_3 & y_3^2 & x_3y_3^2 & y_3^3 & x_3y_3^3 \\
			& & &  & \dots& & &  \\
			1 & x_8 & y_8 & x_8y_8 & y_8^2 & x_8y_8^2 & y_8^3 & x_8y_8^3 
		\end{bmatrix}}_C \begin{bmatrix}
		\alpha_1 \\ \alpha_2 \\ \alpha_3 \\ \dots \\ \alpha_8 
		\end{bmatrix} = \begin{bmatrix}
		\Phi_1 \\ \Phi_2 \\ \Phi_3 \\ \dots \\ \Phi_8
		\end{bmatrix}
		\]
		\[
		C\cdot\alpha=\Phi \Rightarrow \alpha=C^{-1} \cdot \Phi \\
		\]
		\[
		\varphi=P\alpha=P\cdot C^{-1} \cdot \Phi = N\Phi
		\]
		\[
		P=\begin{bmatrix}
			1 & x & y & xy & y^2 & xy^2 & y^3 & xy^3
		\end{bmatrix}, \quad \alpha=\begin{bmatrix}
		\alpha_1 & \dots & \alpha_8
		\end{bmatrix}
		\]
		
		\underline{Самая большая сложность - в составлении $C^{-1}$.}
		
		\newpage
		\begin{center}
			\textbf{Серендипово семейство}
		\end{center}

		\begin{center}
			\begin{tikzpicture}
				\draw[->] (0,0) -- (6,0) node[below right] {$x$};
				\draw[->] (0,0) -- (0,5) node[below left] {$y$};
				\draw (1,1) rectangle ++(4,3);
				\filldraw (1,1) circle (1pt) node[left] {1};
				\filldraw (1,4) circle (1pt) node[left] {4};
				\filldraw (5,1) circle (1pt) node[right] {2};
				\filldraw (5,4) circle (1pt) node[right] {3};
			\end{tikzpicture}
		\end{center}
		
		Линейная аппроксимация: 
		\[
		\varphi = \alpha_1 +\alpha_2 x+\alpha_3 y +\alpha_4 xy
		\]
		\begin{center}
			\begin{tikzpicture}
				\draw[->] (0,2.5) -- (6,2.5) node[below right] {$x$};
				\draw[->] (3,0) -- (3,5) node[left] {$y$};
				\draw (1,1) rectangle ++(4,3);
				\filldraw (1,1) circle (1pt) node[above left] {1} node[below] {$(-b, -a)$};
				\filldraw (1,4) circle (1pt) node[below left] {4} node[above] {$(-b, a)$};
				\filldraw (5,1) circle (1pt) node[above right] {2} node[below] {$(b, -a)$};
				\filldraw (5,4) circle (1pt) node[below right] {3} node[above] {$(b, a)$};
			\end{tikzpicture}
		\end{center}
		\[
			\begin{cases}
				\Phi_1 = \alpha_1 + \alpha_2 \cdot (-b) + \alpha_3 \cdot (-a) + \alpha_4 \cdot ab,\\
				\Phi_2 = \alpha_1 + \alpha_2 \cdot (b) + \alpha_3 \cdot (-a) + \alpha_4 \cdot (-ab),\\
				\Phi_3 = \alpha_1 + \alpha_2 \cdot (b) + \alpha_3 \cdot (a) + \alpha_4 \cdot ab,\\
				\Phi_4 = \alpha_1 + \alpha_2 \cdot (-b) + \alpha_3 \cdot (a) + \alpha_4 \cdot (-ab),\\
			\end{cases}
		\]

		В матричном виде:
		\[
		\begin{bmatrix}
			1 & -b & -a & ab \\
			1 & b & -a &-ab \\
			1 & b & a & ab \\
			1 & -b & a & -ab
		\end{bmatrix}
		\begin{bmatrix}
			\alpha_1 \\ \alpha_2 \\ \alpha_3 \\ \alpha_4 
		\end{bmatrix} = \begin{bmatrix}
		\Phi_1 \\ \Phi_2 \\ \Phi_3 \\ \Phi_4  		
		\end{bmatrix}; \qquad C^{-1}= \begin{bmatrix}
		\frac{1}{4} & \frac{1}{4} & \frac{1}{4} & \frac{1}{4} \\
		-\frac{1}{4b} & \frac{1}{4b} & \frac{1}{4b} & -\frac{1}{4b} \\
		-\frac{1}{4a} & -\frac{1}{4a} & \frac{1}{4a} & \frac{1}{4a} \\
		\frac{1}{4ab} & -\frac{1}{4ab} & \frac{1}{4ab} & -\frac{1}{4ab} 
		\end{bmatrix}
		\]
		\[
	N = P\cdot C^{-1} = \frac{1}{4ab}\begin{bmatrix}
		1 & x & y & xy 
	\end{bmatrix} \begin{bmatrix}
	ab & ab & ab & ab \\
	-a & a & a & -a \\
	-b & -b & b & b \\
	1 & -1 & 1 & -1
	\end{bmatrix} =  \frac{1}{4ab} \begin{bmatrix}
	(b-x)(a-y) \\ (b+x)(a-y) \\ (b+x)(a+y) \\ (b-x)(a+y)
	\end{bmatrix}
		\]
		
	Этот метод не рационален. Поэтому функции формы будем искать с помощью их свойств:
	\begin{center}
		\begin{tikzpicture}
			\draw[->] (0,2.5) -- (6,2.5) node[below right] {$x$};
			\draw[->] (3,0) -- (3,5) node[left] {$y$};
			\draw (1,1) rectangle ++(4,3);
			\filldraw (1,1) circle (1pt) node[above left] {1};
			\filldraw (1,4) circle (1pt) node[below left] {4};
			\filldraw (5,1) circle (1pt) node[above right] {2};
			\filldraw (5,4) circle (1pt) node[below right] {3};
			\draw[red, dashed] (-0.5,1) -- (6.5,1) node[right] {$f_1 = (a + y)$};
			\draw[blue, dashed] (-0.5,4) -- (6.5,4) node[right] {$f_2 = (a - y)$};
			\draw[teal, dashed] (1,5.5) -- (1,-0.5) node[below] {$f_3 = (b + x)$};
			\draw[orange, dashed] (5,5.5) -- (5,-0.5) node[below] {$f_4 = (b - x)$};
			\node at (4.75, 3.25) {\footnotesize $a$};
			\node at (4.75, 1.75) {\footnotesize $a$};
			\node at (2, 0.75) {\footnotesize $b$};
			\node at (4, 0.75) {\footnotesize $b$};
		\end{tikzpicture}
	\end{center}
	\[
		\begin{cases}
			N_1 = \displaystyle \frac{f_2 \cdot f_4}{f_2 \cdot f_4 |_{x=-b,\ y = -a}} = \frac{1}{4ab} (a-y) (b-x)\\
			N_2 = \frac{1}{4ab}(b+x)(a-y)\\
			N_3 = \frac{1}{4ab}(b+x)(a+y)\\
			N_4 = \frac{1}{4ab}(b-x)(a+y)
		\end{cases}
	\]
	\[
		\frac{\partial \varphi}{\partial x} = \alpha_2 + \alpha_4 y,\ 
		\frac{\partial \varphi}{\partial y} = \alpha_3 + \alpha_4 x
	\]

	\begin{center}
		\textbf{Естественная криволинейная система координат}
	\end{center}
	\begin{center}
		\begin{tikzpicture}
			\draw[->] (0,2.5) -- (6,2.5) node[below right] {$\xi$};
			\draw[->] (3,0) -- (3,5) node[left] {$\eta$};
			\draw (1,1) rectangle ++(4,3);
			\filldraw (1,1) circle (1pt) node[above left] {1};
			\filldraw (1,4) circle (1pt) node[below left] {4};
			\filldraw (5,1) circle (1pt) node[above right] {2};
			\filldraw (5,4) circle (1pt) node[below right] {3};
			\draw[red, dashed] (-0.5,1) -- (6.5,1) node[right] {$f_1 = (1 + \eta)$};
			\draw[blue, dashed] (-0.5,4) -- (6.5,4) node[right] {$f_2 = (1 - \eta)$};
			\draw[teal, dashed] (1,5.5) -- (1,-0.5) node[below] {$f_3 = (1 + \xi)$};
			\draw[orange, dashed] (5,5.5) -- (5,-0.5) node[below] {$f_4 = (1 - \xi)$};
			\node at (4.75, 3.25) {\footnotesize $1$};
			\node at (4.75, 1.75) {\footnotesize $1$};
			\node at (2, 0.75) {\footnotesize $1$};
			\node at (4, 0.75) {\footnotesize $1$};
		\end{tikzpicture}
	\end{center}
	\[
		\xi = \frac{x}{b},\ -1\leq \xi \leq 1,\ \ \eta = \frac{y}{a},\ -1\leq \eta \leq 1
	\]
	\[
		\begin{cases}
			N_1 = \frac{1}{4} (1-\xi) (1-\eta)\\
			N_2 = \frac{1}{4} (1+\xi)(1-\eta)\\
			N_3 = \frac{1}{4} (1+\xi)(1+\eta)\\
			N_4 = \frac{1}{4} (1-\xi)(1+\eta)
		\end{cases}
	\]

	В случае четырехугольного элемента более сложной формы естественную систему координат можно задать следующим образом:
	\begin{center}
		\begin{tikzpicture}
			\draw[->, dashed] (-1,2.5) -- (7,2.5) node[below right] {$\xi$};
			\draw[->, dashed] (3,0) -- (3,5) node[left] {$\eta$};
			\draw (1,1) -- (5,1);
			\draw (-0.5,4) -- (6.5,4);
			\draw (1,1) -- (-0.5,4);
			\draw (6.5,4) -- (5,1);
			\draw[dashed, red] (-1.5,1) -- (7.5,1) node[right] {\footnotesize $\eta = -1$};
			\draw[dashed, red] (-1.5,2.5) -- (7.5,2.5) node[right] {\footnotesize $\eta = 0$};
			\draw[dashed, red] (-1.5,4) -- (7.5,4) node[right] {\footnotesize $\eta = 1$};
			\draw[dashed, teal] (3,4.5) -- (3,0) node[below] {\footnotesize $\xi = 0$};
			\draw[dashed, teal] (1.5,0) -- (-1,5) node[above] {\footnotesize $\xi = -1$};
			\draw[dashed, teal] (4.5,0) -- (7,5) node[above] {\footnotesize $\xi = 1$};
			\draw[dashed, teal] (2.25,0) -- (1,5) node[above] {\footnotesize $\displaystyle \xi = -\frac{1}{2}$};
			\draw[dashed, teal] (3.75,0) -- (5,5) node[above] {\footnotesize $\displaystyle \xi = \frac{1}{2}$};
		\end{tikzpicture}
	\end{center}
\end{document}