\documentclass{bmstu}

\usepackage{mathtools}

\usepackage{physics}
\usepackage{pdfpages}
\usepackage{tabularx}
\usepackage{longtable}
\usepackage{xfrac}
\usepackage{amssymb}
\usepackage{dsfont}
\usepackage{upgreek}
\usepackage{color, colortbl}
\usepackage{listings}
\usepackage{ amssymb }
\usepackage{tikz}
\usetikzlibrary{decorations.markings} % Для корректной работы маркировок и стрелок

\begin{document}
\section*{11/10}
	
		\begin{center}
		\textbf{Двумерные краевые задачи}
	\end{center}
	\begin{equation} \label{2d_eq}
	-\frac{\partial}{\partial x} \left( K \frac{\partial u}{\partial x} \right) -\frac{\partial}{\partial y} \left( K \frac{\partial u}{\partial y} \right) + bu = f
	\end{equation}
	
	$K(x, y),\ b(x, y),\ f(x,y)$ -- заданные (гладкие) функции.
	
	$u(x, y)$ -- неизвестная.
	
	$\Omega$ -- область, где задано уравнение \eqref{2d_eq}, $\Gamma$ -- граница $\Omega$ .
	
	\begin{tikzpicture}
    % Оси
    \draw[->] (-1, 0) -- (3, 0) node[right] {};
    \draw[->] (0, -1) -- (0, 3) node[above] {};
    
    % Область Омега
    \draw[thick] (0.5, 0.5) .. controls (1.2, 1.8) and (2, 1.5) .. (1.5, 2.5) 
    .. controls (0.5, 3) and (-0.5, 2) .. (0, 1) -- cycle;
    \node at (0.8, 1.2) {$\Omega$};
    
    % Вектор n
    \draw[->] (1.5, 2.5) -- (1.9, 2.9);
    \node at (2, 2.8) {$\vec{n}$};
    
    % Граница Gamma1
    \draw[decorate,decoration={markings,mark=at position 0.5 with {\arrow{>}}}] 
        (1.5, 2.5) arc[start angle=40, end angle=210, radius=2cm];
    \node at (-1, 1) {$\Gamma_1$};
    
    % Граница Gamma2
    \draw[decorate,decoration={markings,mark=at position 0.5 with {\arrow{>}}}] 
        (0, 1) arc[start angle=190, end angle=330, radius=1cm];
    \node at (1, 0) {$\Gamma_2$};

    % Точка O и вектор ksi
    \node at (-0.2, -0.2) {$O$};
    \draw[->] (0, 0) -- (0.5, 0.5);
    \node at (0.3, 0.3) {$\vec{\xi}$};
\end{tikzpicture}
	
	Замкнутый контур $\Gamma$ -- гладкий, за исключением конечного числа угловых точек, в которых внутренний угол $\alpha \in [0; \pi]$.
	\[\Gamma = \underbrace{\Gamma_1}_{\text{I рода}} \cup \underbrace{\Gamma_2}_{\text{II/III рода}}\]
	\begin{center}
		$u(\xi) = \hat u(\xi)$ -- на $\Gamma_1$ (заданное значение).
	\end{center}
	\[K(\xi)\frac{\partial u}{\partial n} = \hat \sigma (\xi) - \text{ на } \Gamma_2\]
	\[\frac{\partial u}{\partial n} = \frac{\partial u}{\partial x} l_x + \frac{\partial u}{\partial y} l_y,\ 
	\begin{cases}
	l_x = \cos(\alpha x) = \cos(\vec x, \vec n),\\
	l_y = \cos(\alpha y) = \cos(\vec y, \vec n),\\
	||\vec n|| = 1
	\end{cases}\]

	\begin{tikzpicture}
    % Оси
    \draw[->] (0, 0) -- (4, 0) node[right] {$x$}; % Ось X
    \draw[->] (0, 0) -- (0, 4) node[above] {$y$}; % Ось Y

    % Кривая
    \draw[thick] plot [domain=1:3] (\x, {2/\x}); % Кривая y = 2/x

    % Точка (x, y)
    \fill (2, 1) circle (2pt); % Точка на кривой
    \node[below left] at (2, 1) {$(x, y)$};

    % Линии dx и dy
    \draw[dashed] (2, 1) -- (3, 1) node[midway, below] {$dx$}; % dx
    \draw[dashed] (2, 1) -- (2, 2) node[midway, right] {$dy$}; % dy

    % Обозначения
    \node at (3.2, 1) {$l_x$};
    \node at (2, 2.2) {$l_y$};
\end{tikzpicture}
	
	Составим невязку:
	\[r(x, y) = -\frac{\partial}{\partial x} \left( K \frac{\partial u}{\partial x} \right) -\frac{\partial}{\partial y} \left( K \frac{\partial u}{\partial y} \right) + bu - f = 0\]
	
	\[\iint\limits_{\Omega} r(x, y) \cdot v \, dx \, dy = 0,\]
	
	где $v(x, y)$ -- пробная (гладкая) функция; на $\Gamma_1$: $v = 0$
	
	\begin{equation}\label{iint}
	\iint\limits_{\Omega} \left [ -\frac{\partial}{\partial x} \left( K \frac{\partial u}{\partial x} \right) -\frac{\partial}{\partial y} \left( K \frac{\partial u}{\partial y} \right) + bu - f \right ] \cdot v \, dx \, dy = 0
	\end{equation}
	
	Рассмотрим первый интеграл:
	\[-\iint\limits_{\Omega} \frac{\partial}{\partial x} \left( K \frac{\partial u}{\partial x} \right) \cdot v \, dx \, dy\]
	
	Формула Гаусса-Остроградского:
	\[\iint\limits_{\Omega} \frac{\partial}{\partial x} F dx dy = \int \limits_{\Gamma} F \cdot l_x d \xi\]
	
	Представим $F$ в виде $F = uv$:
	\[\frac{\partial}{\partial x} F = \frac{\partial}{\partial x} (uv) = \frac{\partial u}{\partial x} v + u \frac{\partial v}{\partial x}\ \Rightarrow\ 
	\frac{\partial u}{\partial x} v = \frac{\partial}{\partial x} (uv) - u \frac{\partial v}{\partial x}\]
	
	Отсюда:
	\[\iint\limits_{\Omega} \frac{\partial u}{\partial x} v \, dx \, dy = \iint\limits_{\Omega} \left( \frac{\partial}{\partial x} (uv) - u \frac{\partial v}{\partial x} \right) \, dx \, dy\]
	
	И по формуле Гаусса-Остроградского:
	\[\iint\limits_{\Omega} \frac{\partial u}{\partial x} v \, dx \, dy = \int \limits_{\Gamma} uv \cdot l_x d \xi - \iint\limits_{\Omega} \frac{\partial v}{\partial x} u \, dx \, dy\]
	
	Тогда:
	\[-\iint\limits_{\Omega} \frac{\partial}{\partial x} \left( K \frac{\partial u}{\partial x} \right) \cdot v \, dx \, dy = 
	-\int \limits_{\Gamma} K \cdot \frac{\partial u}{\partial x} \cdot v \cdot l_x d \xi + \iint\limits_{\Omega} \frac{\partial v}{\partial x} \left(  K \cdot \frac{\partial u}{\partial x} \right) \, dx \, dy\]
	\[-\iint\limits_{\Omega} \frac{\partial}{\partial y} \left( K \frac{\partial u}{\partial y} \right) \cdot v \, dx \, dy = 
	-\int \limits_{\Gamma} K \cdot \frac{\partial u}{\partial y} \cdot v \cdot l_y d \xi + \iint\limits_{\Omega} \frac{\partial v}{\partial y} \left(  K \cdot \frac{\partial u}{\partial y} \right) \, dx \, dy\]
	
	Подставим в \eqref{iint}:
	\[\iint\limits_{\Omega} \left [ \frac{\partial v}{\partial x} \cdot K \cdot \frac{\partial u}{\partial x} + \frac{\partial v}{\partial y} \cdot K \cdot \frac{\partial u}{\partial y} + bu \cdot v - f  \cdot v \right ] \, dx \, dy -\]
	\[-\int \limits_{\Gamma_2} K \cdot v \cdot \left(\frac{\partial u}{\partial x} \cdot l_x + \frac{\partial u}{\partial y} \cdot l_y \right) = 0\]

	\underline{Задача упругости}
 	$ u(x, y) $ - перемещения \\
 	\[ \varepsilon = [\varepsilon_x \quad \varepsilon_y]^T, \quad \varepsilon_x = - \frac{\partial u}{\partial y},\quad \varepsilon_y = - \frac{\partial u}{\partial y} \]
	\[ \varepsilon = Lu,\quad L = \left[- \frac{\partial}{\partial x} \quad - \frac{\partial}{\partial y}\right]^T \]
	\[ \sigma = k \varepsilon = kLu \]
	%тут рисунок
	\[ d \Omega = dxdy \]
	\[ \sigma_x dy + \sigma_y dx + (f-bu) dxdy = \left(\sigma_x + \frac{\partial \sigma_x}{\partial x}dx\right) dy + \left(\sigma_y + \frac{\partial \sigma_y}{\partial y}dy\right) dx \]
	\[ \frac{\partial \sigma_x}{\partial x} + \frac{\partial \sigma_y}{\partial y} = f - bu \]
	\[ L_* = \left[\frac{\partial}{\partial x} \quad \frac{\partial}{\partial y}\right],\quad L_* = -L^T \]
	\[ L_* \sigma = L_*kLu = f - bu \]
	\[ \left[\frac{\partial}{\partial x} \quad \frac{\partial}{\partial y}\right] \cdot k \cdot 
	\begin{bmatrix}
		-\frac{\partial u}{\partial x}\\
		- \frac{\partial u}{\partial y}
	\end{bmatrix}
 	= f-bu \]
	\[-\frac{\partial}{\partial x} \left ( K \frac{\partial u}{\partial x} \right) - \frac{\partial}{\partial y} \left ( K \frac{\partial u}{\partial y} \right) = f \cdot bu \Leftrightarrow \eqref{2d_eq}\]
	
	%тут рисунок
	\[\Gamma_1: u(\xi) = \hat u(\xi)\]
	\[dx = \cos \alpha_y d\xi = l_y d\xi,\ \ dy = \cos \alpha_x d\xi = l_x d\xi\]
	\[\sigma_y dx + \sigma_x dy + \hat \sigma d\xi = 0\]
	\[\sigma_x = K \varepsilon_x = -K \frac{\partial u}{\partial x},\ \sigma_y = K \varepsilon_y = -K \frac{\partial u}{\partial y}\]
	
	Тогда:
	\[-K \frac{\partial u}{\partial y} dx -K \frac{\partial u}{\partial x} dy + \hat \sigma d\xi = 0\]
	\[-K \frac{\partial u}{\partial y} l_y d\xi -K \frac{\partial u}{\partial x} l_x d\xi + \hat \sigma d\xi = 0\]
	\[K \frac{\partial u}{\partial n} = \hat \sigma\]
	
	Отсюда:
	\begin{equation}\label{2d_eq_2}
	\iint \limits_{\Omega} \left(\frac{\partial v}{\partial x} K \frac{\partial u}{\partial x} + \frac{\partial v}{\partial y} K \frac{\partial u}{\partial y} + bu \cdot v - f \cdot v\right) dx dy - \\
	\int \limits_{\Gamma} \underbrace{ \left(K \frac{\partial u}{\partial x} l_x + K \frac{\partial u}{\partial y} l_y\right)}_{K \cdot \frac{\partial u}{\partial n} = \hat \sigma(\xi)}  \cdot v d\xi = 0 
	\end{equation}
	
	\[\iint \limits_{\Omega} ((Lu)^T K (Lu) + v^T bu - v^T f) dxdy - \int \limits_{\Gamma_2} v^T \hat \sigma d\xi = 0\]
	
	   \end{document}