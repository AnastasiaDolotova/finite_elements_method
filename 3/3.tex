\documentclass{bmstu}


\usepackage{physics}
\usepackage{pdfpages}
\usepackage{tabularx}
\usepackage{longtable}
\usepackage{xfrac}
\usepackage{amssymb}
\usepackage{dsfont}
\usepackage{upgreek}
\usepackage{color, colortbl}
\usepackage{ amssymb }
\usepackage{listings}
\usepackage{tikz}

\graphicspath{
	{graphics/}
}

\begin{document}
	
	\section*{20/09}
	
	\begin{center}
		\textbf{Хранение разреженных матриц}
	\end{center}
	
		\underline{Опр.} Портретом разреженной матрицы называется множество:
		\[
		P_A=\{(i,j):a_{ij}\neq0\}
		\]
		
		Возможны следующие случаи:
		\begin{enumerate}
			\item Матрица несимметрична и ее портрет несимметричен:
			\[
			\exists \ (i, j) \in P_A: (j, i) \notin P_A \Leftrightarrow A\neq A^{\text{Т}}
			\]
			\[
			\begin{bmatrix}
				1&2 \\ 0&0
			\end{bmatrix}
			\Rightarrow 
			\begin{bmatrix}
				\ast&\ast \\ 0&0
			\end{bmatrix}
			\]
			\item Матрица несимметрична но её портрет симметричен:
			\[
			A\neq A^{\text{Т}}, \text{ но } (i, j) \in P_A, (j, i) \in P_A, a_{ij}\neq a_{ij}
			\]
			\[
			\begin{bmatrix}
				1&0 \\ 0&2
			\end{bmatrix}
			\Rightarrow 
			\begin{bmatrix}
				\ast&0 \\ 0&\ast
			\end{bmatrix}
			\]
			\item Если матрица симметрична ($a_{ji}=a_{ij}$), то её портрет симметричен:
			\[
			\begin{bmatrix}
				1&0 \\ 0&1
			\end{bmatrix}
			\Rightarrow 
			\begin{bmatrix}
				\ast&0 \\ 0&\ast
			\end{bmatrix}
			\]
		\end{enumerate}
		
Форматы хранения разреженных матриц:
		
\underline{CSR} - compressed sparse row

\underline{CSC} - compressed sparse column

\underline{CSLR} - compressed sparse low-triangle row \\

Рассмотрим формат CSR:
\begin{enumerate}
	\item aelem - массив, хранящий все $a_{ij}\neq 0$ в строках
	\item jptr - массив размерности aelem, указывает номер $N_j$ элемента $a_{ij}$
	\item iptr - массив размерности $n+1$ ($n$ - размерность СЛАУ), хранит число элементов $a_{ij}\neq 0$ в строке
	
	iptr$[i+1]$ - iptr$[i]$ - число элементов в i-ой строке
	
	iptr$[n+1]$ - число элементов в aelem + 1
	%\item adiag - все диагональные элементы
	%\item altr - элементы нижнего треугольника
\end{enumerate}

\underline{Пример:}
\[
A=
\begin{bmatrix}
	9&0&0&3&1&0&1 \\
	0&11&2&1&0&0&2 \\
	0&1&10&2&0&0&0 \\
	2&1&2&9&1&0&0 \\
	1&0&0&1&12&0&1 \\
	0&0&0&0&0&8&0 \\
	2&2&0&0&3&0&8 
\end{bmatrix}
\]

Матрица $A$ несимметрична, а $P_A$ симметричен, т.к. если $a_{ij}\neq 0$, то $a_{ji}\neq 0$.

aelem: $\left[\left[ 9, 3, 1, 1 \right], \left[11, 2, 1, 2\right], \left[1, 10, 2\right],\left[2, 1, 2 ,9, 1\right],\left[1, 1, 12, 1\right],\left[8\right],\left[2, 2, 3, 8\right] \right]$.

jptr: $\left[\left[1, 4, 5, 7\right], \left[2, 3, 4, 7\right], \left[2, 3, 4\right],\left[1, 2, 3, 4, 5\right],\left[1, 4, 5, 7 \right],\left[6\right],\left[1, 2, 5, 7\right] \right]$.

iptr: $\left[1, 5, 9, 12, 17, 21, 22, 26\right]$. \\

Рассмотрим формат CSLR:

Обычно такой формат используется для хранения симметричных матриц, т.е. в памяти хранятся только элементы верхнего (или нижнего) треугольника.
Если матрица несимметрична, то храним ещё и элементы нижнего треугольника, но уже по столбцам (для сохранения структуры).

В нашем примере:

adiag: $\left[9, 11, 10, 9, 12, 8, 8\right]$ -- все диагональные элементы.

altr: $\left[1,2, 1, 2, 1, 1, 2, 2, 3\right]$ -- элементы нижнего треугольника.

autr: $\left[2, 3, 1, 2, 1, 1, 1, 2, 1\right]$ -- элементы верхнего треугольника.

jptr: $\left[2, 1, 2, 3, 1, 4, 1, 2, 5\right]$.

iptr: $\left[1, 1, 1, 2, 5, 7, 7, 10\right]$.


Входные данные: $\overline{x}$, aelem, jptr, iptr, n.
\[
z = A\overline{x}, \ A - CSR
\]

\newpage
\textbf{\textit{Листинг 1 - алгоритм составления вектора $z$}}
\begin{lstlisting}
i: 1,.., n 
{
	z[i]=0;
	j: iptr[i],..,iptr[i+1]-1 
	{
		z[i]=z[i]+x[jptr[j]]*aelem[j];
	}
}
\end{lstlisting}

\newpage
	\begin{center}
	\textbf{Учет граничных условий}
\end{center}

\underline{Граничные условия I рода}

Зададим температуру в узле $a_{33}$:
\[
A=\begin{bmatrix}
	\cdots & &\cdots & &\cdots  \\
	\cdots & 0 & 1 & 0 & \cdots \\
	\cdots & &\cdots & &\cdots 
\end{bmatrix}
\]

Чтобы не испортилась симметричность портрета, искусственно обнулим необходимые элементы массива aelem.  \\

\textbf{\textit{Листинг 2 - алгоритм задания граничных условий}}
\begin{lstlisting}
	i = iptr[k],..,iptr[k+1]-1
	{ 
		if jptr[i]=k:
			aelem[jptr[k]]=1;
		else:
			aelem[jptr[k]]=0;
	}
\end{lstlisting}

\underline{Метод Холецкого}
\[ \begin{matrix}
	A\approx LU\\A = LU+R
\end{matrix}
\]

где $L$ - нижнетреугольная матрица;\\
\hangindent=2.05cm
$U$ - верхнетреугольная матрица;\\
	$R$ - ошибка округления.
	
	\[
	\begin{cases}
		Ax= b \\ LUx= b \\ Ly = b \Rightarrow Ux=y
	\end{cases}
	\]
	
	\newpage
	\begin{center}
		\textbf{Предобусловливание}
	\end{center}
	
	Число обусловленности: $\frac{\lambda_{max}}{\lambda_{min}}$
	
	Если число обусловленности велико, система плохо обусловлена, и небольшие ошибки в данных могут привести к большим ошибкам в решении. Если число обусловленности близко к 1, система хорошо обусловлена, и решения будут стабильными.
	
	Рассмотрим СЛАУ $Ax = b$, где $A$ -- плохо обусловленная матрица. 
	
	Пусть $M$ -- невырожденная матрица размерности $n \times n$. Домножим СЛАУ на матрицу, обратную $M$:
	\[M^{-1}Ax = M^{-1}b\]
	\[A^*x = b^*\]
	
	M - матрица предобусловливания
	
	\begin{enumerate}
		\item M должна быть по возможности близка к A (пример: M = diag(A)).
		\item M должна быть легко вычислима.
		\item M должна быть легко обратима.
	\end{enumerate}
\end{document}

















