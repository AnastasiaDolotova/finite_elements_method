\documentclass[a4paper, 12pt]{article}

\usepackage[left=2cm,right=2cm,
    top=2cm,bottom=2cm,bindingoffset=0cm]{geometry}

\usepackage[T2A]{fontenc}
\usepackage[utf8]{inputenc}
\usepackage{color}
\usepackage{graphicx}
\usepackage{caption}
\usepackage{subcaption}
\usepackage{tikz}
\usepackage[english, russian]{babel}
\usepackage{amsmath,amsfonts,amssymb,amsthm,mathtools}

\begin{document}


 \begin{figure}[h] 
  \begin{center}
 \begin{tikzpicture}
  \def\xmax{4}
    \def\ymax{4}

    \foreach \x in {0,1,...,\xmax} {
        \foreach \y in {0,1,...,\ymax} {
            \draw[gray] (\x, 0) -- (\x, \ymax);
            \draw[gray] (0, \y) -- (\xmax, \y);
        }
    }

 \foreach \x in {0,1,...,3} {
        \foreach \y in {0,1,...,3 } {
            \draw[gray] (\x, \y + 1) -- (\x + 1, \y);
        }
    }

 \foreach \x in {0,1,...,\xmax} {
        \foreach \y in {0,1,...,\ymax} {
            \fill (\x,\y) circle (2pt);
        }
    }
    
    \foreach \x in {1,2,...,5} {
   	 \node at (\x - 1.2, -0.4) {\x};
    }
    
    \foreach \x in {6,7,...,10} {
   	 \node at (\x - 6.2, 1.4) {\x};
    }
    
    	\draw (0.3,0.3) circle (2.5mm); 
   	 \node at (0.3,0.3) {1}; 
	 
	\draw (1.3,0.3) circle (2.5mm); 
   	 \node at (1.3,0.3) {3}; 
	 
	\draw (0.7,0.7) circle (2.5mm); 
   	 \node at (0.7,0.7) {2}; 
	 
	\draw (1.7,0.7) circle (2.5mm); 
   	 \node at (1.7,0.7) {4}; 
	 
	 \draw (3.7,3.7) circle (2.5mm); 
   	 \node at (3.7,3.7) {n}; 
 
 \end{tikzpicture}
  \end{center}
 \caption{Пример дискретизации области}
 \end{figure}

  \begin{figure}[h] 
  \begin{center}
 \begin{tikzpicture}[scale=1.5]
 	 \fill[orange!30] (0,0,1) -- (2,0,1) -- (2,0.5,1) -- (0,0.5,1) -- cycle;
	 \fill[orange!30] (2,0,1) -- (2,0.5,1) -- (2,0.5,0) -- (2,0,0) -- cycle;
	 
	 \fill[green!30] (0,0.5,1) -- (2,0.5,1) -- (2,1,1) -- (0,1,1) -- cycle;
	 \fill[green!30] (2,0.5,1) -- (2,1,1) -- (2,1,0) -- (2,0.5,0) -- cycle;
	 \fill[green!30] (0,1,1) -- (2,1,1) -- (2,1,0) -- (0,1,0) -- cycle;
	 
  
        \draw[thick] (0,0,1) -- (2,0,1) -- (2,1,1) -- (0,1,1) -- cycle; 

        \draw[thick] (2,0,0) -- (2,0,1); 
        \draw[thick] (2,1,0) -- (2,1,1); 
        \draw[thick] (0,1,0) -- (0,1,1); 
	\draw[thick] (2,0,0) -- (2,1,0); 
	\draw[thick] (0,1,0) -- (2,1,0); 
	
	\draw[thick] (0,0.5,1) -- (2,0.5,1); 
	 \draw[thick] (2,0.5,0) -- (2,0.5,1);

	\draw[black, ultra thick] (0,0,1) rectangle (0.5,0.5,1);
	\draw[black, ultra thick] (0,0.5,1) rectangle (0.5,1,1);
	\draw[ultra thick] (0,0.5,1) -- (0.5,0,1);
	\draw[ultra thick] (0,1,1) -- (0.5,0.5,1);

 
 \end{tikzpicture}
  \end{center}
 \caption{Пример трехмерной области, состоящей из двух материалов}
 \end{figure}


  \begin{figure}[h] 
  \begin{center}
 \begin{tikzpicture}
 	 
	 \draw[thick] (0,0) arc[start angle=180, end angle=270, radius=2cm];
	 \draw[thick] (2,-2) -- (0,-2);
	 \draw[thick] (0,-2) -- (0,0);
	 
	  \draw[thick] (7,-2) -- (5,-2) -- (5,0)  -- cycle; 
	  
	  \draw[->, thick] (2.5,-1) -- (4,-1) ;
	\node at (3.2,-0.5) {J};	
	
	\node at (-1,-1) {$x,y,z$};
	\node at (7,-1) {$u,v,w$};
	
	\fill (2,-2) circle (2pt);
	\fill (1,-2) circle (2pt);
	\fill (0,-2) circle (2pt);
	\fill (0,-1) circle (2pt);
	\fill (0,0) circle (2pt);
	\fill (0.58579, -1.42) circle (2pt);
	
	\fill (7,-2) circle (2pt);
	\fill (6,-2) circle (2pt);
	\fill (5,-2) circle (2pt);
	\fill (5,-1) circle (2pt);
	\fill (5,0) circle (2pt);
	\fill (6, -1) circle (2pt);

 
 \end{tikzpicture}
  \end{center}
 \caption{Приведение криволинейного элемента}
 \end{figure}

  \begin{figure}[h] 
  \begin{center}
 \begin{tikzpicture}[scale=0.7]
 	 
	  \draw[thick] (0,0) -- (5,0) -- (2,3)  -- cycle; 
	  \draw[thick] (0,0) -- (3.33333, 1.66667);
	  \draw[thick] (2,3) -- (2,1);
	  
	  \fill (0,0) circle (2pt);
	  \fill (5,0) circle (2pt);
	  \fill (2,3) circle (2pt);
	  \fill (3.33333, 1.66667) circle (2pt);
	  \fill (2,1) circle (2pt);
	  
	  \node at (1.5,1.3) {1};
	  \node at (2.5,1.7) {2};
	  \node at (3,0.5) {3};

 
 \end{tikzpicture}
  \end{center}
 \caption{Пример плохой дискретизации}
 \end{figure}

  \begin{figure}[h] 
  \begin{center}
 \begin{tikzpicture}[domain = -1:0] 
 	 \foreach \x in {0,1,...,3} {
        \foreach \y in {0,1,...,3} {
            \draw[gray] (\x, 0) -- (\x, 3);
            \draw[gray] (0, \y) -- (3, \y);
        }
    }
    
     \foreach \x in {0,1,...,2} {
        \foreach \y in {0,1,...,2 } {
            \draw[gray] (\x, \y + 1) -- (\x + 1, \y);
        }
    }
    
    \foreach \x in {0,1,...,2} {
        \foreach \y in {0,1,...,2} {
            \draw[gray] (\x+3, 0) -- (\x+3, 2);
            \draw[gray] (3, \y) -- (5, \y);
        }
    }
    
    \foreach \x in {0,1} {
        \foreach \y in {0,1} {
            \draw[gray] (\x+3, \y + 1) -- (\x + 4, \y);
        }
    }
    
    \draw[gray] plot (\x+4, \x * \x+2);
    
    \foreach \x in {1,2,...,4} {
   	 \node at (\x - 1.2, 3.3) {\x};
    }
    
    \foreach \x in {9,10,...,12} {
   	 \node at (\x - 9.2, 2.3) {\x};
    }
    
    \foreach \x in {13,14,...,17} {
   	 \node at (\x - 13.2, 1.3) {\x};
    }
    
     \foreach \x in {18,19,...,22} {
   	 \node at (\x - 18.2, 0.3) {\x};
    }
    
     \node at (3.8, 2.3) {5};
     
     \foreach \x in {6,7,...,8} {
     	\node at (4.8, 8.3 - \x) {\x};
	}
	
	\node at (4.3, 0.3) {$\bullet$};
	\node at (2.5,-1) {$\bullet\ 8,\ 17,\ 22$};
	\node at (2.5,-1.8) {$R_\bullet = 14$};

%second pic
     \foreach \x in {7,8,...,10} {
        \foreach \y in {0,1,...,3} {
            \draw[gray] (\x, 0) -- (\x, 3);
            \draw[gray] (7, \y) -- (10, \y);
        }
    }
    
     \foreach \x in {0,1,...,2} {
        \foreach \y in {0,1,...,2 } {
            \draw[gray] (\x+7, \y + 1) -- (\x + 8, \y);
        }
    }
    
    \foreach \x in {7,8,...,9} {
        \foreach \y in {0,1,...,2} {
            \draw[gray] (\x+3, 0) -- (\x+3, 2);
            \draw[gray] (10, \y) -- (12, \y);
        }
    }
    
     \foreach \x in {0,1} {
        \foreach \y in {0,1} {
            \draw[gray] (\x+10, \y + 1) -- (\x + 11, \y);
        }
    }
    
    \draw[gray] plot (\x+11, \x * \x+2);
        
     \foreach \x in {1,2,...,4} {
   	 \node at (6.8, -0.7+\x) {\x};
    }
    
     \foreach \x in {5,6,...,8} {
   	 \node at (7.8, -4.7+\x) {\x};
    }
    
     \foreach \x in {9,10,...,12} {
   	 \node at (8.8, -8.7+\x) {\x};
    }
    
     \foreach \x in {13,14,...,16} {
   	 \node at (9.8, -12.7+\x) {\x};
    }
    
    \foreach \x in {17,18,...,19} {
   	 \node at (10.8, -16.7+\x) {\x};
    }
    
    \foreach \x in {20,21,...,22} {
   	 \node at (11.8, -19.7+\x) {\x};
    }
    
    	\node at (11.3, 0.3) {$\bullet$};
	\node at (9.5,-1) {$\bullet\ 20,\ 18,\ 17$};
	\node at (9.5,-1.8) {$R_\bullet = 3$};
    
    
    \draw[red, thick] (0,4) circle (2.5mm);
    \draw[red, thick] (-0.15, 4-0.15) -- (0.15, 4+0.15);
    \draw[red, thick] (-0.15, 4+0.15) -- (0.15, 4-0.15);

\draw[green, thick] (7,4) circle (2.5mm);
\draw[green, thick] (6.9, 4+0.1) -- (7, 4-0.1) -- (7.3, 4+0.3);

 
 \end{tikzpicture}
  \end{center}
 \caption{Пример правильной и неправильной нумерации узлов}
 \end{figure}


  \begin{figure}[h] 
  \begin{center}
 \begin{tikzpicture}
 	 
	  \draw[thick] (0,0) -- (1,1) -- (3,1) -- (2,0)  -- cycle; 
	  \draw[thick] (0,0) -- (3,1);
	 
	 \draw[thick] (5,0) -- (6,1) -- (8,1) -- (7,0)  -- cycle; 
	 \draw[thick] (7,0) -- (6,1);
	 
	 \node at (-0.4,-0.4) {i};
	 \node at (2,-0.4) {j};
	 \node at (3.4,1) {k};
	 
	 \node at (4.8,-0.4) {1};
	 \node at (7,-0.4) {2};
	 \node at (5.5,1) {3};
	 \node at (8.4,1) {4};
	 
	 \draw[red, thick] (0,1.5) circle (2.5mm);
    \draw[red, thick] (-0.15, 1.5-0.15) -- (0.15, 1.5+0.15);
    \draw[red, thick] (-0.15, 1.5+0.15) -- (0.15, 1.5-0.15);

\draw[green, thick] (5,1.5) circle (2.5mm);
\draw[green, thick] (4.9, 1.5+0.1) -- (5, 1.5-0.1) -- (5.3, 1.5+0.3);

	\draw (6,0.4) circle (2.5mm); 
   	 \node at (6,0.4) {1}; 
	 
	\draw (7,0.6) circle (2.5mm); 
   	 \node at (7,0.6)  {2}; 
 
 \end{tikzpicture}
  \end{center}
 \caption{Пример плохой и хорошей дискретизации}
 \end{figure}






















\end{document}